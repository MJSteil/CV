\documentclass[color=TUDa-1b]{komacv}
% ********* Setup ********* %
\usepackage{ifthen}
\newboolean{german}
\setboolean{german}{false} % Set to true for German, false for English
\definecolor{TUDa-1b}{cmyk}{1.0,0.6,0.0,0.0}
\KOMAoptions{fontsize=10pt}

\newcommand{\ger}[1]{\ifthenelse{\boolean{german}}{#1}{}}
\newcommand{\eng}[1]{\ifthenelse{\boolean{german}}{}{#1}}
\newcommand{\lang}[2]{\ifthenelse{\boolean{german}}{#1}{#2}}
\newcommand{\mmyyyy}[2]{\lang{#1.#2}{#1/#2}}

\usepackage[% 
	backend=biber,
	hyperref=true, % use hyperref for links (DOI, arXiv, URLs)
	natbib=true, % natbib provides functionalities related to numeric-comp
	isbn=false, % do not show ISBN numbers
	style=numeric-comp, % numeric citation scheme including compressions, e.g., [8, 3, 1, 7, 2] => [1–3, 7, 8]
	sorting=none, % sort by citation
	giveninits=true, % only render given name initials
	maxbibnames=3, % use et al. for more than three authors
	date=year, % use year only 
	urldate=iso, % use iso-date, e.g, 2021/01/01, for urldates
	seconds=true, % print seconds when specified (urldate=iso wants this option)
	defernumbers=true
]{biblatex}

% ************ Citing *********** %
%[https://tex.stackexchange.com/a/399170/148765]
\NewBibliographyString{refname}
\NewBibliographyString{refsname}
\DefineBibliographyStrings{english}{%
  refname = {Ref\adddot},
  refsname = {Refs\adddot}
}

\DeclareCiteCommand{\ccite}
  {%
  \ifnum\thecitetotal=1
    \bibstring{refname}%
  \else%
    \bibstring{refsname}%
  \fi%
  \addspace\bibopenbracket%
  \usebibmacro{cite:init}%
   \usebibmacro{prenote}}
  {\usebibmacro{citeindex}%
   \usebibmacro{cite:comp}}
  {}
  {\usebibmacro{cite:dump}%
   \usebibmacro{postnote}%
   \bibclosebracket}

\newrobustcmd*{\Ccite}{\bibsentence\ccite}

% ************ Citation commands ************ %
\preto\fullcite{\AtNextCite{\defcounter{maxnames}{99}}} % Render up to 99 names when using \fullcite

% ************ Bibliography style adjustments ************ %
% Font sizes
\renewcommand*{\bibfont}{\footnotesize} % general font for bibliography
\newcommand{\bibfontsmall}{\scriptsize} % smaller font for bibliography (used in URLs, DOIs, arXiv Links)

% Redefine bibstrings
\DefineBibliographyStrings{english}{%
	urlseen = {accessed on},
}
	
% General formatting
\DeclareFieldFormat{journaltitle}{\mkbibemph{#1}\isdot} % title in italic
\DeclareFieldFormat[article,online,unpublished,software,misc]{title}{\mkbibquote{#1\isdot}} % title in quotes
\DeclareFieldFormat[thesis]{title}{\mkbibemph{#1\isdot}} % title in italic
\DeclareFieldFormat[article]{volume}{\textbf{#1}} % bold volume
\DeclareFieldFormat[article]{pages}{#1} % show page range instead of first page (\mkfirstpage[]{#1})
\renewcommand*{\newunitpunct}{\addcomma\space} % use comma as seperator instead of dots


% New bibmacro for article driver to display journal+volume+date+pages (e.g. Nature 59 (1898) 200–200)
\makeatletter
\newbibmacro*{journal+volume+date+pages}{%
	\usebibmacro{journal}%
	\setunit*{\addspace}%
	\iffieldundef{series}%
		{}%
		{\newunit\printfield{series}\setunit{\addspace}}%
	\printfield{volume}%
	\setunit{\addspace}%
	\printtext[parens]{\usebibmacro{date}}
	\setunit*{\addspace}%
	\printfield{pages}%
	\newunit
}
\makeatother

% New bibmacro for article driver to display epring+url [compare standard.bbx, line 901]
\makeatletter
\newbibmacro*{eprint+url}{%
	\iftoggle{bbx:eprint}
		{\usebibmacro{eprint}}
		{}%
	\newunit\newblock
	\iftoggle{bbx:url}
		{\usebibmacro{url+urldate}}
		{}
}
\makeatother

% et al. in italic [modifies biblatex.def, line 1157]
\renewbibmacro*{name:andothers}{%
	\ifboolexpr{
		test {\ifnumequal{\value{listcount}}{\value{liststop}}}
		and
		test \ifmorenames
	}{
		\ifnumgreater{\value{liststop}}{1}
		{\finalandcomma}
		{}%
		\printdelim{andothersdelim}\bibstring[\emph]{andothers}
	}{}
}

% Use new variable (font size) to format url [modifies biblatex.def, line 582]
\DeclareFieldFormat{url}{\mkbibacro{URL}\addcolon\space{\bibfontsmall\url{#1}}}

% Use new variable (font size) to format url [modifies biblatex.def, line 495]
\DeclareFieldFormat{doi}{%
	\mkbibacro{DOI}\addcolon\space
	\ifhyperref
		{\bibfontsmall\href{https://doi.org/#1}{\nolinkurl{#1}}}
		{\bibfontsmall\nolinkurl{#1}}
}

% Use new variable (font size) to format url [modifies biblatex.def, line 521]
\makeatletter
\DeclareFieldFormat{eprint:arxiv}{%
	arXiv\addcolon\space
	\ifhyperref
		{\bibfontsmall\href{https://arxiv.org/\abx@arxivpath/#1}{%
			\nolinkurl{#1}%
			\iffieldundef{eprintclass}
			{}
			{\bibfontsmall\addspace\texttt{\mkbibbrackets{\thefield{eprintclass}}}}}%
		}
		{%
			\nolinkurl{#1}%
			\iffieldundef{eprintclass}
			{}
			{\bibfontsmall\addspace\texttt{\mkbibbrackets{\thefield{eprintclass}}}}%
		}%
}
\makeatother

% Add parentheses around block [modifies standard.bbx, line 851]
\renewbibmacro*{publisher+location+date}{%
	\nopunct
	\printtext[parens]{%
		\printlist{publisher}%
		\setunit*{\addcomma\space}%
		\printlist{location}%
		\setunit*{\addcomma\space}%
		\usebibmacro{date}%
	}%
}

% Custom driver for article [modifies standard.bbx, line 28]
\DeclareBibliographyDriver{article}{%
	\usebibmacro{bibindex}%
	\usebibmacro{begentry}%
	\usebibmacro{author/translator+others}%
	\setunit{\printdelim{nametitledelim}}\newblock%
	\usebibmacro{title}%
	\newunit%
	\printlist{language}%
	\newunit\newblock%
	\usebibmacro{byauthor}%
	\newunit\newblock%
	\usebibmacro{bytranslator+others}%
	\newunit\newblock%
	\printfield{version}%
	\newunit\newblock%
	\addcomma%
	\iffieldundef{doi}{\usebibmacro{journal+volume+date+pages}}{%
	\ifhyperref%
		{\href{https://doi.org/\thefield{doi}}{\usebibmacro{journal+volume+date+pages}}}%
		{\usebibmacro{journal+volume+date+pages}}%
	}{}%	
	\newunit%
	\usebibmacro{byeditor+others}%
	\newunit\newblock
	\iftoggle{bbx:isbn}
		{\printfield{issn}}
		{}%
	\newunit\newblock
	\usebibmacro{eprint+url}%
	\newunit\newblock%
	\printfield{note}%
	\newunit\newblock
	\usebibmacro{addendum+pubstate}%
	\setunit{\bibpagerefpunct}\newblock
	\usebibmacro{pageref}%
	\newunit\newblock
	\iftoggle{bbx:related}
		{\usebibmacro{related:init}%
		 \usebibmacro{related}}
		{}%
	\usebibmacro{finentry}
}

% Custom driver for book [modifies standard.bbx, line 66]
\DeclareBibliographyDriver{book}{%
\usebibmacro{bibindex}%
	\usebibmacro{begentry}%
	\usebibmacro{author/translator+others}%
	\setunit{\labelnamepunct}\newblock
	\usebibmacro{maintitle+title}%
	\iffieldundef{edition}{\setunit{\addspace}}{\addcomma\space \printfield{edition}}%
	\newunit\newblock
	\printlist{language}%
	\newunit\newblock
	\usebibmacro{byauthor}%
	\newunit\newblock
	\usebibmacro{byeditor+others}%
	\newunit
	\printfield{volumes}%
	\newunit\newblock
	\usebibmacro{series+number}%
	\newunit\newblock
	\iftoggle{bbx:isbn}
		{\printfield{isbn}}
		{}%
	\newunit\newblock
	\usebibmacro{publisher+location+date}%
	\newunit\newblock
	\printfield{note}%
	\newunit\newblock
	\usebibmacro{chapter+pages}%
	\newunit\newblock
	\printfield{pagetotal}%
	\newunit\newblock
	\usebibmacro{doi+eprint+url}%
	\newunit\newblock
	\usebibmacro{addendum+pubstate}%
	\setunit{\bibpagerefpunct}\newblock
	\usebibmacro{pageref}%
	\newunit\newblock
	\iftoggle{bbx:related}
		{\usebibmacro{related:init}%
		 \usebibmacro{related}}
		{}%
	\usebibmacro{finentry}%
}

% Custom driver for misc [modifies standard.bbx, line 434]
\DeclareBibliographyDriver{misc}{%
	\usebibmacro{bibindex}%
	\usebibmacro{begentry}%
	\usebibmacro{author/editor+others/translator+others}%
	\setunit{\printdelim{nametitledelim}}\newblock
	\usebibmacro{title}%
	\newunit
	\printlist{language}%
	\newunit\newblock
	\usebibmacro{byauthor}%
	\newunit\newblock
	\usebibmacro{byeditor+others}%
	\newunit\newblock
	\printfield{howpublished}%
	\newunit\newblock
	\printfield{type}%
	\newunit
	\printfield{version}%
	\newunit\newblock
	\usebibmacro{journal}
	\nopunct
	\newunit\newblock
	\printtext[parens]{\usebibmacro{organization+location+date}}%
	\newunit\newblock
	\usebibmacro{doi+eprint+url}%
	\newunit\newblock
	\printfield{note}%
	\newunit\newblock
	\usebibmacro{addendum+pubstate}%
	\setunit{\bibpagerefpunct}\newblock
	\usebibmacro{pageref}%
	\newunit\newblock
	\iftoggle{bbx:related}
		{\usebibmacro{related:init}%
		 \usebibmacro{related}}
		{}%
	\usebibmacro{finentry}%
}

% Custom driver for unpublished [modifies standard.bbx, line 695]
\DeclareBibliographyDriver{unpublished}{%
	\usebibmacro{bibindex}%
	\usebibmacro{begentry}%
	\usebibmacro{author}%
	\setunit{\printdelim{nametitledelim}}\newblock
	\usebibmacro{title}%
	\newunit
	\printlist{language}%
	\newunit\newblock
	\usebibmacro{byauthor}%
	\newunit\newblock
	\printfield{howpublished}%
	\newunit\newblock
	\printfield{type}%
	\newunit\newblock
	\usebibmacro{event+venue+date}%
	\newunit\newblock
	\usebibmacro{location+date}%
	\newunit\newblock
	\usebibmacro{doi+eprint+url}%
	\newunit\newblock
	\printfield{note}%
	\newunit\newblock
	\usebibmacro{addendum+pubstate}%
	\setunit{\bibpagerefpunct}\newblock
	\usebibmacro{pageref}%
	\newunit\newblock
	\iftoggle{bbx:related}
		{\usebibmacro{related:init}%
		 \usebibmacro{related}}
		{}%
	\usebibmacro{finentry}%
}


\bibliography{MJSteil_CV_2024}

\usepackage[utf8]{inputenc}
\lang{\usepackage[english, main=ngerman]{babel}}{\usepackage[english]{babel}}
\usepackage{csquotes}
\usepackage{enumitem}

\pagestyle{scrheadings}
\clearpairofpagestyles

\ifoot{\lang{Lebenslauf}{CV} -- \firstname~\familyname}
\ofoot{\pagemark /\,\usekomafont{pagenumber}{\pageref*{LastPage}}}

\renewcommand*{\familydefault}{\sfdefault}

\makeatletter\renewcommand*{\cventry}[7][\@afterelementsvspace]{%
	\cvitem[#1]{#2}{%
		{\bfseries#3}%
		\ifstrempty{#4}{}{, {\itshape#4}}%
		\ifstrempty{#5}{}{, #5}%
		\ifstrempty{#6}{}{, #6}%
		\ifx%
			&#7&%
		\else{%
			\newline{}\begin{minipage}[t]{\linewidth}%
				\small#7%
			\end{minipage}%
		}%
		\fi%
	}%cvitem
}%cventry
\makeatother

\makeatletter\newcommand*{\programmingLanguage}[4]{%
	\textbf{#1:} %
	\ifstrequal{#2}{0}{\lang{Anfänger}{Beginner} }{%
	\ifstrequal{#2}{1}{\lang{Fortgeschritten}{Intermediate} }{%
	\ifstrequal{#2}{2}{\lang{Sehr Fortgeschritten}{Advanced} }{%
	\ifstrequal{#2}{3}{\lang{Experte}{Expert} }{%
	}}}}%
	(#3 \lang{Jahr}{year}\ifstrequal{#3}{1}{}{\lang{e}{s}})\newline\vspace{3pt}%
	\ifstrempty{#4}{}{%
		\begin{itemize}[topsep=0pt]
		\setlength\itemsep{-.1em}\small
			#4
		\end{itemize}
	}\vspace{6pt}%
}%
\makeatother

\renewcommand*\urlbordercolor{TUDa-1b}
\hypersetup{
	colorlinks=true,
	citecolor=TUDa-1b,
	linkcolor=TUDa-1b,
	urlcolor=TUDa-1b
}

\lang{\renewcommand*\pdftitle{Lebenslauf -- Martin Jakob Steil}}{\renewcommand*\pdftitle{Curriculum vitae -- Martin Jakob Steil}}

% ********* Daten ********* %
\renewcommand*\pdfauthor{Martin Jakob Steil}

\renewcommand*{\title}{CV}% PDF metadata
\renewcommand*{\acadtitle}{Dr.~rer.~nat.}
\renewcommand*{\firstname}{Martin~Jakob~Steil}

\renewcommand*{\github}{\hspace{-2pt}\href{https://github.com/MJSteil}{github.com/MJSteil}}
\renewcommand*{\linkedin}{\hspace{-2pt}\href{https://www.linkedin.com/in/martin-jakob-steil}{martin-jakob-steil}}
\renewcommand*{\extrainfo}{\faOrcid{}~\href{https://orcid.org/0000-0001-8465-9803}{0000-0001-8465-9803}} %\faOrcid from fontawesome5 package might require modification of komacv.cls to include \RequirePackage{fontawesome5} instead of fontawesome

% Photo
\setlength\fboxrule{0pt}
\setlength\mframepicshift{0.25cm}
\photo[frame]{3cm}{steil_2024}

% Private data -- NOT FOR GITHUB
%\renewcommand*{\addressstreet}{}
%\renewcommand*{\addresscity}{}
%\renewcommand*{\address}[2]{\addressstreet{#1}\addresscity{#2}}
%\renewcommand*{\mobile}{}
%\renewcommand*{\email}{}
\newcommand{\gradePhD}{}
\newcommand{\gradeMSc}{}
\newcommand{\gradeBSc}{}
\newcommand{\gradeAbi}{}

% Unused fields
% \renewcommand*{\cvquote}{\enquote{...}} %p.373
%\renewcommand*{\familyname}{}
%\renewcommand*{\phonenr}{}
%\renewcommand*{\faxnr}{}
%\renewcommand*{\homepage}{}
% \renewcommand*{\twitter}{}

% ********* CV ********* %
\begin{document}
\raggedbottom
\maketitle


% ****** Personal data ****** %
\lang{\section{Persönliche Daten}}{\section{Personal information}}

\cvitem{\lang{Geburtsdatum}{Date of birth}}{\lang{23. Oktober}{October 23,} 1991}
\cvitem{\lang{Geburtsort}{Place of birth}}{Offenbach am Main}
\cvitem{\lang{Nationalität}{Nationality}}{\lang{Deutsch}{German}}


% ****** Professional experience ****** %
\lang{\section{Berufserfahrung}}{\section{Professional experience}}

\cventry%
	{\mmyyyy{08}{2017}--\mmyyyy{12}{2021}}%
	{\lang{Wissenschaftlicher Mitarbeiter}{Research associate}}%
	{\lang{Institut für Kernphysik, Technische Universität Darmstadt}{Institute for Nuclear Physics, Technische Universität Darmstadt}}%
	{}{}%
	{\lang%
		{Studie von inhomogenen chirale Kondensaten mit der funktionalen Renormierungsgruppe (26 Wochenarbeitsstunden)}%
		{Study of inhomogeneous chiral condensates within the functional renormalization group (26h/week)}
	}
\cventry%
	{\mmyyyy{04}{2016}--\mmyyyy{09}{2016} \mmyyyy{10}{2011}--\mmyyyy{10}{2015}}%
	{\lang{Studentische Hilfskraft}{Student assistant}}{\lang{Zentrale Studienberatung, Technische Universität Darmstadt}{Central student advisory office, Technische Universität Darmstadt}}%
	{}{}%
	{\lang%
		{IT-Support, Webmaster, Print- und Webdesign (15 Wochenarbeitsstunden)}%
		{IT support, webmaster, print- and web-design (15h/week)}
	}


% ****** Education ****** %
\lang{\section{Ausbildung}}{\section{Education}}

\cventry%
	{\mmyyyy{08}{2017}--\mmyyyy{06}{2024}}%
	{\lang{Promotion in Physik}{PhD in Physics}}{Technische Universität Darmstadt}%
	{}{}%
	{\lang%
		{Dissertation ``From zero-dimensional theories to inhomogeneous phases with the functional renormalization group'' unter der Betreuung von Priv.-Doz. Dr. Michael Buballa}%
		{Dissertation ``From zero-dimensional theories to inhomogeneous phases with the functional renormalization group'' under the supervision of Priv.-Doz. Dr. Michael Buballa}\\
		\gradePhD{}
	}
\cventry%
	{\mmyyyy{02}{2015}--\mmyyyy{07}{2017}}%
	{Master of Science in \lang{Physik}{Physics}}%
	{Technische Universität Darmstadt}%
	{}{}%
	{\lang%
		{Masterarbeit ``Structure of slowly rotating magnetized neutron stars in a perturbative approach'' unter der Betreuung von Priv.-Doz. Dr. Michael Buballa}%
		{Master’s thesis ``Structure of slowly rotating magnetized neutron stars in a perturbative approach'' under the supervision of Priv.-Doz. Dr. Michael Buballa}\\
		\gradeMSc{}
	}
\cventry%
	{\mmyyyy{10}{2011}--\mmyyyy{02}{2015}}%
	{Bachelor of Science in \lang{Physik}{Physics}}%
	{Technische Universität Darmstadt}%
	{}{}%
	{\lang%
		{Bachelorarbeit ``Hadron-quark crossover and massive hybrid stars'' unter der Betreuung von Priv.-Doz. Dr. Michael Buballa}%
		{Bachelor’s thesis ``Hadron-quark crossover and massive hybrid stars'' under the supervision of Priv.-Doz. Dr. Michael Buballa}\\
		\gradeBSc{}
	}
\cventry%
	{\mmyyyy{05}{2011}}%
	{Abitur}%
	{Claus-von-Stauffenberg-Schule Rodgau}
	{}{}%
	{\gradeAbi{}}


% ****** Research profile ****** %
\lang{\section{Forschungsprofil}}{\section{Research profile}}

\lang{\subsection{Dissertation}}{\subsection{Dissertation}}
\cvitem{\mmyyyy{08}{2024}}{\fullcite{Steil:2024phd}}

\lang{\subsection{Veröffentlichungen}}{\subsection{Publications}}
\cvitem{\mmyyyy{09}{2022}}{\fullcite{Koenigstein:2021syz}}%0d part I paper -- zerod1
\cvitem{\mmyyyy{09}{2022}}{\fullcite{Koenigstein:2021rxj}} %0d part II paper -- zerod2
\cvitem{\mmyyyy{09}{2022}}{\fullcite{Steil:2021cbu}} %0d part III paper -- zerod3
\cvitem{\mmyyyy{08}{2022}}{\fullcite{Koenigstein:2021llr}}\bigskip % GN stability analysis paper-- gns
\cvitem{\mmyyyy{08}{2021}}{\fullcite{Stoll:2021ori}} %GN published manuskript/preprint -- gn

\lang{\subsection{Lehre}}{\subsection{Teaching}}
\cventry%
	{\mmyyyy{08}{2017}--\mmyyyy{12}{2021}}%
	{\lang{Lehrassistent}{Teaching assistant}}%
	{\lang{Fachbereich Physik, Technische Universität Darmstadt}{Department of Physics, Technische Universität Darmstadt}}%
	{}{}%
	{\begin{itemize}
		\setlength\itemsep{-.1em}
			\item \textit{\lang{``Quantenfeldtheorie II'' (Wintersemester 2019/20)}{``Quantum Field Theory II'' (winter term 2019/20)}}
			\item \textit{\lang{``Quantenfeldtheorie I'' (Sommersemester 2019)}{``Quantum Field Theory I'' (summer term 2019)}}
			\item \textit{\lang{``Klassische Teilchen und Felder für Lehramt'' (Wintersemester 2018/19)}{``Classical Particles and Fields for Teachers'' (winter term 2018/19)}}
			\item \textit{\lang{``Klassische Teilchen und Felder für Lehramt'' (Wintersemester 2017/18)}{``Classical Particles and Fields for Teachers'' (winter term 2017/18)}}
	\end{itemize}}
\cventry%
	{\mmyyyy{08}{2017}--\mmyyyy{12}{2021}}%
	{\lang{Mitbetreuung von Abschlussarbeiten}{Supervision of theses}}%
	{\lang{Fachbereich Physik, Technische Universität Darmstadt}{Department of Physics, Technische Universität Darmstadt}}%
	{}{}%
	{\lang{Zweitkorrektur und Mitbetreuung von zwei Bachelor Arbeiten}{Second referee and co-supervision of two bachelor theses}}


% ****** Languages ****** %
\lang{\section{Sprachen}}{\section{Languages}}

\cvitemwithcomment{\lang{Deutsch}{German}}{\lang{fließend}{fluent}}{\lang{Muttersprache}{mother tongue}}
\cvitemwithcomment{\lang{Englisch}{English}}{\lang{fließend}{fluent}}{\lang{mündlich und schriftlich}{oral and written}}


% ****** Skills and qualifications ****** %
\lang{\section{Kenntnisse und Qualifikationen}}{\section{Skills and qualifications}}

\cvitem%
	{\lang{Theoretische Hochenergiephysik}{Theoretical high-energy physics}}%
	{\lang%
		{Funktionale Renormierungsgruppe, nulldimensionale Theorien, stark wechselwirkende Systeme, (in)homogene chirale Phasen und statistische Physik}%
		{Functional renormalization group, zero-dimensional theories, strongly interacting systems, (in)homogeneoues phases, and statistical physics}
	}
\cvitem%
	{\lang{Mathematik}{Math}}%
	{\lang%
		{(Numerische) Strömungsmechanik, gewöhnliche und partielle Differentialgleichungen, lokale und globale Minimierung, hochdimensionale Integrale und funktionale Methoden}%
		{(Numerical) fluid dynamics, ordinary and partial differential equations, local and global minimization, high-dimensional integrals, and functional methods}
	}
\cvitem%
	{\lang{Programmieren}{Programming}}%
	{Wolfram Language/Mathematica, C/C++, Python, Doxygen\lang{ und}{, and} Git}
\cvitem%
	{\lang{Schriftsatz}{Typesetting}}%
	{\LaTeX, Microsoft Office\lang{ und}{, and} Adobe InDesign}
\cvitem%
	{\lang{Wissenschaftliche Visualisierung}{Scientific visualization}}%
	{Mathematica, Matplotlib, Axodraw2\lang{ und}{, and} TikZ}
\cvitem%
	{\lang{Grafikdesign}{Graphics design}}%
	{Adobe Photoshop\lang{ und}{ and} Maxon Cinema 4D}
\cvitem%
	{\lang{Soft-Skill-Training}{Soft skill training}}%
	{%
		HGS-HIRe \lang{Soft-Skill-Training-Programm (2018-2020)}{soft skill training program (2018-2020)}:\newline
		\begin{itemize}
		\setlength\itemsep{-.1em}
			\item Basic Course 1: \textit{Making an Impact as an Effective Researcher}
			\item Basic Course 2: \textit{Leading Teams in a Research Environment}
			\item Basic Course 3: \textit{Career and Leadership Development}
		\end{itemize}
	}

\cvitem%
	{\lang{Führerschein}{Drivers license}}%
	{\lang{Klasse B seit \mmyyyy{10}{2008}}{German driving license class B since \mmyyyy{10}{2008}}}


% ****** Academic associations ****** %
\lang{\section{Akademische Zugehörigkeiten}}{\section{Academic associations}}

\cvitem%
	{\mmyyyy{08}{2017}--\mmyyyy{10}{2024}}%
	{\lang%
		{Junior-Mitglied des \textit{Sonderforschungsbereichs TransRegio 211} \newline (gefördert durch die Deutsche Forschungsgemeinschaft)}%
		{Junior member of the \textit{Collaborative Research Center TransRegio 211}\newline (funded by Deutsche Forschungsgemeinschaft)}
	}
\cvitem%
	{\mmyyyy{01}{2018}--\mmyyyy{07}{2024}}%
	{\lang%
		{Mitglied der \textit{Helmholtz Graduate School for Hadron and Ion Research} (HGS-HIRe)}
		{Member of the \textit{Helmholtz Graduate School for Hadron and Ion Research} (HGS-HIRe)}
	}
\cvitem%
	{\lang{seit}{since} \mmyyyy{08}{2017}}%
	{\lang%
		{Mitglied der \textit{Deutschen Physikalischen Gesellschaft e. V.}}%
		{Member of the \textit{Deutsche Physikalische Gesellschaft e. V.}}
	}


% ****** Awards ****** %
\lang{\section{Auszeichnungen}}{\section{Awards}}

\cvitem%
	{\mmyyyy{10}{2019}}%
	{\lang%
		{Giersch-Excellence-Grant: in Anerkennung herausragender Leistungen im Doktorarbeitsprojekt \textit{``Inhomogeneous Chiral Condensates within the Functional Renormalization Group''}}
		{Giersch-Excellence-Grant: in recognition of outstanding achievements in the doctoral thesis project \textit{``Inhomogeneous Chiral Condensates within the Functional Renormalization Group''}}
	}
\cvitem%
	{\mmyyyy{05}{2011}}
	{\lang%
		{GDCh-Preis für den besten Abiturienten in Chemie an der Claus-von-Stauffenberg-Schule Rodgau}
		{GDCh-Award for the best high-school graduate in chemistry at the Claus-von-Stauffenberg-Schule Rodgau}
	}


% ****** Awards ****** %
\lang{\section{Interessen}}{\section{Interests}}

\cvitem%
	{\lang{Computer}{Computer}}%
	{\lang{Case-Modding, Wasserkühlung und Programmieren}{Case modding, watercooling, and programming}}
\cvitem%
	{\lang{Sport}{Sports}}%
	{\lang{Schwimmen, online Schach und Sudoku}{Swimming, online chess, and Sudoku}}
\cvitem%
	{\lang{Theoretische Physik}{Theoretical physics}}
	{\lang%
		{Funktionale Renormierungsgruppe, nulldimensionale Theorien, stark wechselwirkende Systeme und (in)homogene chirale Phasen}%
		{Functional Renormalizationgroup, zero-dimensional theories, strongly interacting systems, in(homogeneous) phases}
	}
\cvitem%
	{\lang{Künstliche Intelligenz und Machine Learning}{Artificial intelligence and machine learning}}
	{\lang%
		{KI Art (Dall-E, Bing), KI Musik (Suno), Github Copilot und ChatGPT-Anwendungen}%
		{AI Art (Dall-E, Bing), AI Musik (Suno), Github Copilot und ChatGPT-Applications}
	}

\end{document}
